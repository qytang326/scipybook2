
% Default to the notebook output style

    


% Inherit from the specified cell style.




    
\documentclass[11pt]{article}

    
    
    \usepackage[T1]{fontenc}
    % Nicer default font (+ math font) than Computer Modern for most use cases
    \usepackage{mathpazo}

    % Basic figure setup, for now with no caption control since it's done
    % automatically by Pandoc (which extracts ![](path) syntax from Markdown).
    \usepackage{graphicx}
    % We will generate all images so they have a width \maxwidth. This means
    % that they will get their normal width if they fit onto the page, but
    % are scaled down if they would overflow the margins.
    \makeatletter
    \def\maxwidth{\ifdim\Gin@nat@width>\linewidth\linewidth
    \else\Gin@nat@width\fi}
    \makeatother
    \let\Oldincludegraphics\includegraphics
    % Set max figure width to be 80% of text width, for now hardcoded.
    \renewcommand{\includegraphics}[1]{\Oldincludegraphics[width=.8\maxwidth]{#1}}
    % Ensure that by default, figures have no caption (until we provide a
    % proper Figure object with a Caption API and a way to capture that
    % in the conversion process - todo).
    \usepackage{caption}
    \DeclareCaptionLabelFormat{nolabel}{}
    \captionsetup{labelformat=nolabel}

    \usepackage{adjustbox} % Used to constrain images to a maximum size 
    \usepackage{xcolor} % Allow colors to be defined
    \usepackage{enumerate} % Needed for markdown enumerations to work
    \usepackage{geometry} % Used to adjust the document margins
    \usepackage{amsmath} % Equations
    \usepackage{amssymb} % Equations
    \usepackage{textcomp} % defines textquotesingle
    % Hack from http://tex.stackexchange.com/a/47451/13684:
    \AtBeginDocument{%
        \def\PYZsq{\textquotesingle}% Upright quotes in Pygmentized code
    }
    \usepackage{upquote} % Upright quotes for verbatim code
    \usepackage{eurosym} % defines \euro
    \usepackage[mathletters]{ucs} % Extended unicode (utf-8) support
    \usepackage[utf8x]{inputenc} % Allow utf-8 characters in the tex document
    \usepackage{fancyvrb} % verbatim replacement that allows latex
    \usepackage{grffile} % extends the file name processing of package graphics 
                         % to support a larger range 
    % The hyperref package gives us a pdf with properly built
    % internal navigation ('pdf bookmarks' for the table of contents,
    % internal cross-reference links, web links for URLs, etc.)
    \usepackage{hyperref}
    \usepackage{longtable} % longtable support required by pandoc >1.10
    \usepackage{booktabs}  % table support for pandoc > 1.12.2
    \usepackage[inline]{enumitem} % IRkernel/repr support (it uses the enumerate* environment)
    \usepackage[normalem]{ulem} % ulem is needed to support strikethroughs (\sout)
                                % normalem makes italics be italics, not underlines
    

    
    
    % Colors for the hyperref package
    \definecolor{urlcolor}{rgb}{0,.145,.698}
    \definecolor{linkcolor}{rgb}{.71,0.21,0.01}
    \definecolor{citecolor}{rgb}{.12,.54,.11}

    % ANSI colors
    \definecolor{ansi-black}{HTML}{3E424D}
    \definecolor{ansi-black-intense}{HTML}{282C36}
    \definecolor{ansi-red}{HTML}{E75C58}
    \definecolor{ansi-red-intense}{HTML}{B22B31}
    \definecolor{ansi-green}{HTML}{00A250}
    \definecolor{ansi-green-intense}{HTML}{007427}
    \definecolor{ansi-yellow}{HTML}{DDB62B}
    \definecolor{ansi-yellow-intense}{HTML}{B27D12}
    \definecolor{ansi-blue}{HTML}{208FFB}
    \definecolor{ansi-blue-intense}{HTML}{0065CA}
    \definecolor{ansi-magenta}{HTML}{D160C4}
    \definecolor{ansi-magenta-intense}{HTML}{A03196}
    \definecolor{ansi-cyan}{HTML}{60C6C8}
    \definecolor{ansi-cyan-intense}{HTML}{258F8F}
    \definecolor{ansi-white}{HTML}{C5C1B4}
    \definecolor{ansi-white-intense}{HTML}{A1A6B2}

    % commands and environments needed by pandoc snippets
    % extracted from the output of `pandoc -s`
    \providecommand{\tightlist}{%
      \setlength{\itemsep}{0pt}\setlength{\parskip}{0pt}}
    \DefineVerbatimEnvironment{Highlighting}{Verbatim}{commandchars=\\\{\}}
    % Add ',fontsize=\small' for more characters per line
    \newenvironment{Shaded}{}{}
    \newcommand{\KeywordTok}[1]{\textcolor[rgb]{0.00,0.44,0.13}{\textbf{{#1}}}}
    \newcommand{\DataTypeTok}[1]{\textcolor[rgb]{0.56,0.13,0.00}{{#1}}}
    \newcommand{\DecValTok}[1]{\textcolor[rgb]{0.25,0.63,0.44}{{#1}}}
    \newcommand{\BaseNTok}[1]{\textcolor[rgb]{0.25,0.63,0.44}{{#1}}}
    \newcommand{\FloatTok}[1]{\textcolor[rgb]{0.25,0.63,0.44}{{#1}}}
    \newcommand{\CharTok}[1]{\textcolor[rgb]{0.25,0.44,0.63}{{#1}}}
    \newcommand{\StringTok}[1]{\textcolor[rgb]{0.25,0.44,0.63}{{#1}}}
    \newcommand{\CommentTok}[1]{\textcolor[rgb]{0.38,0.63,0.69}{\textit{{#1}}}}
    \newcommand{\OtherTok}[1]{\textcolor[rgb]{0.00,0.44,0.13}{{#1}}}
    \newcommand{\AlertTok}[1]{\textcolor[rgb]{1.00,0.00,0.00}{\textbf{{#1}}}}
    \newcommand{\FunctionTok}[1]{\textcolor[rgb]{0.02,0.16,0.49}{{#1}}}
    \newcommand{\RegionMarkerTok}[1]{{#1}}
    \newcommand{\ErrorTok}[1]{\textcolor[rgb]{1.00,0.00,0.00}{\textbf{{#1}}}}
    \newcommand{\NormalTok}[1]{{#1}}
    
    % Additional commands for more recent versions of Pandoc
    \newcommand{\ConstantTok}[1]{\textcolor[rgb]{0.53,0.00,0.00}{{#1}}}
    \newcommand{\SpecialCharTok}[1]{\textcolor[rgb]{0.25,0.44,0.63}{{#1}}}
    \newcommand{\VerbatimStringTok}[1]{\textcolor[rgb]{0.25,0.44,0.63}{{#1}}}
    \newcommand{\SpecialStringTok}[1]{\textcolor[rgb]{0.73,0.40,0.53}{{#1}}}
    \newcommand{\ImportTok}[1]{{#1}}
    \newcommand{\DocumentationTok}[1]{\textcolor[rgb]{0.73,0.13,0.13}{\textit{{#1}}}}
    \newcommand{\AnnotationTok}[1]{\textcolor[rgb]{0.38,0.63,0.69}{\textbf{\textit{{#1}}}}}
    \newcommand{\CommentVarTok}[1]{\textcolor[rgb]{0.38,0.63,0.69}{\textbf{\textit{{#1}}}}}
    \newcommand{\VariableTok}[1]{\textcolor[rgb]{0.10,0.09,0.49}{{#1}}}
    \newcommand{\ControlFlowTok}[1]{\textcolor[rgb]{0.00,0.44,0.13}{\textbf{{#1}}}}
    \newcommand{\OperatorTok}[1]{\textcolor[rgb]{0.40,0.40,0.40}{{#1}}}
    \newcommand{\BuiltInTok}[1]{{#1}}
    \newcommand{\ExtensionTok}[1]{{#1}}
    \newcommand{\PreprocessorTok}[1]{\textcolor[rgb]{0.74,0.48,0.00}{{#1}}}
    \newcommand{\AttributeTok}[1]{\textcolor[rgb]{0.49,0.56,0.16}{{#1}}}
    \newcommand{\InformationTok}[1]{\textcolor[rgb]{0.38,0.63,0.69}{\textbf{\textit{{#1}}}}}
    \newcommand{\WarningTok}[1]{\textcolor[rgb]{0.38,0.63,0.69}{\textbf{\textit{{#1}}}}}
    
    
    % Define a nice break command that doesn't care if a line doesn't already
    % exist.
    \def\br{\hspace*{\fill} \\* }
    % Math Jax compatability definitions
    \def\gt{>}
    \def\lt{<}
    % Document parameters
    \title{Index}
    
    
    

    % Pygments definitions
    
\makeatletter
\def\PY@reset{\let\PY@it=\relax \let\PY@bf=\relax%
    \let\PY@ul=\relax \let\PY@tc=\relax%
    \let\PY@bc=\relax \let\PY@ff=\relax}
\def\PY@tok#1{\csname PY@tok@#1\endcsname}
\def\PY@toks#1+{\ifx\relax#1\empty\else%
    \PY@tok{#1}\expandafter\PY@toks\fi}
\def\PY@do#1{\PY@bc{\PY@tc{\PY@ul{%
    \PY@it{\PY@bf{\PY@ff{#1}}}}}}}
\def\PY#1#2{\PY@reset\PY@toks#1+\relax+\PY@do{#2}}

\expandafter\def\csname PY@tok@nd\endcsname{\def\PY@tc##1{\textcolor[rgb]{0.67,0.13,1.00}{##1}}}
\expandafter\def\csname PY@tok@sb\endcsname{\def\PY@tc##1{\textcolor[rgb]{0.73,0.13,0.13}{##1}}}
\expandafter\def\csname PY@tok@nl\endcsname{\def\PY@tc##1{\textcolor[rgb]{0.63,0.63,0.00}{##1}}}
\expandafter\def\csname PY@tok@no\endcsname{\def\PY@tc##1{\textcolor[rgb]{0.53,0.00,0.00}{##1}}}
\expandafter\def\csname PY@tok@mo\endcsname{\def\PY@tc##1{\textcolor[rgb]{0.40,0.40,0.40}{##1}}}
\expandafter\def\csname PY@tok@go\endcsname{\def\PY@tc##1{\textcolor[rgb]{0.53,0.53,0.53}{##1}}}
\expandafter\def\csname PY@tok@ge\endcsname{\let\PY@it=\textit}
\expandafter\def\csname PY@tok@gi\endcsname{\def\PY@tc##1{\textcolor[rgb]{0.00,0.63,0.00}{##1}}}
\expandafter\def\csname PY@tok@vc\endcsname{\def\PY@tc##1{\textcolor[rgb]{0.10,0.09,0.49}{##1}}}
\expandafter\def\csname PY@tok@mh\endcsname{\def\PY@tc##1{\textcolor[rgb]{0.40,0.40,0.40}{##1}}}
\expandafter\def\csname PY@tok@nt\endcsname{\let\PY@bf=\textbf\def\PY@tc##1{\textcolor[rgb]{0.00,0.50,0.00}{##1}}}
\expandafter\def\csname PY@tok@vi\endcsname{\def\PY@tc##1{\textcolor[rgb]{0.10,0.09,0.49}{##1}}}
\expandafter\def\csname PY@tok@cpf\endcsname{\let\PY@it=\textit\def\PY@tc##1{\textcolor[rgb]{0.25,0.50,0.50}{##1}}}
\expandafter\def\csname PY@tok@nc\endcsname{\let\PY@bf=\textbf\def\PY@tc##1{\textcolor[rgb]{0.00,0.00,1.00}{##1}}}
\expandafter\def\csname PY@tok@nf\endcsname{\def\PY@tc##1{\textcolor[rgb]{0.00,0.00,1.00}{##1}}}
\expandafter\def\csname PY@tok@ne\endcsname{\let\PY@bf=\textbf\def\PY@tc##1{\textcolor[rgb]{0.82,0.25,0.23}{##1}}}
\expandafter\def\csname PY@tok@mb\endcsname{\def\PY@tc##1{\textcolor[rgb]{0.40,0.40,0.40}{##1}}}
\expandafter\def\csname PY@tok@nn\endcsname{\let\PY@bf=\textbf\def\PY@tc##1{\textcolor[rgb]{0.00,0.00,1.00}{##1}}}
\expandafter\def\csname PY@tok@mf\endcsname{\def\PY@tc##1{\textcolor[rgb]{0.40,0.40,0.40}{##1}}}
\expandafter\def\csname PY@tok@sa\endcsname{\def\PY@tc##1{\textcolor[rgb]{0.73,0.13,0.13}{##1}}}
\expandafter\def\csname PY@tok@cs\endcsname{\let\PY@it=\textit\def\PY@tc##1{\textcolor[rgb]{0.25,0.50,0.50}{##1}}}
\expandafter\def\csname PY@tok@nb\endcsname{\def\PY@tc##1{\textcolor[rgb]{0.00,0.50,0.00}{##1}}}
\expandafter\def\csname PY@tok@sc\endcsname{\def\PY@tc##1{\textcolor[rgb]{0.73,0.13,0.13}{##1}}}
\expandafter\def\csname PY@tok@mi\endcsname{\def\PY@tc##1{\textcolor[rgb]{0.40,0.40,0.40}{##1}}}
\expandafter\def\csname PY@tok@ow\endcsname{\let\PY@bf=\textbf\def\PY@tc##1{\textcolor[rgb]{0.67,0.13,1.00}{##1}}}
\expandafter\def\csname PY@tok@vm\endcsname{\def\PY@tc##1{\textcolor[rgb]{0.10,0.09,0.49}{##1}}}
\expandafter\def\csname PY@tok@fm\endcsname{\def\PY@tc##1{\textcolor[rgb]{0.00,0.00,1.00}{##1}}}
\expandafter\def\csname PY@tok@cm\endcsname{\let\PY@it=\textit\def\PY@tc##1{\textcolor[rgb]{0.25,0.50,0.50}{##1}}}
\expandafter\def\csname PY@tok@sr\endcsname{\def\PY@tc##1{\textcolor[rgb]{0.73,0.40,0.53}{##1}}}
\expandafter\def\csname PY@tok@gt\endcsname{\def\PY@tc##1{\textcolor[rgb]{0.00,0.27,0.87}{##1}}}
\expandafter\def\csname PY@tok@sh\endcsname{\def\PY@tc##1{\textcolor[rgb]{0.73,0.13,0.13}{##1}}}
\expandafter\def\csname PY@tok@gr\endcsname{\def\PY@tc##1{\textcolor[rgb]{1.00,0.00,0.00}{##1}}}
\expandafter\def\csname PY@tok@kd\endcsname{\let\PY@bf=\textbf\def\PY@tc##1{\textcolor[rgb]{0.00,0.50,0.00}{##1}}}
\expandafter\def\csname PY@tok@gh\endcsname{\let\PY@bf=\textbf\def\PY@tc##1{\textcolor[rgb]{0.00,0.00,0.50}{##1}}}
\expandafter\def\csname PY@tok@il\endcsname{\def\PY@tc##1{\textcolor[rgb]{0.40,0.40,0.40}{##1}}}
\expandafter\def\csname PY@tok@ch\endcsname{\let\PY@it=\textit\def\PY@tc##1{\textcolor[rgb]{0.25,0.50,0.50}{##1}}}
\expandafter\def\csname PY@tok@w\endcsname{\def\PY@tc##1{\textcolor[rgb]{0.73,0.73,0.73}{##1}}}
\expandafter\def\csname PY@tok@s1\endcsname{\def\PY@tc##1{\textcolor[rgb]{0.73,0.13,0.13}{##1}}}
\expandafter\def\csname PY@tok@err\endcsname{\def\PY@bc##1{\setlength{\fboxsep}{0pt}\fcolorbox[rgb]{1.00,0.00,0.00}{1,1,1}{\strut ##1}}}
\expandafter\def\csname PY@tok@c1\endcsname{\let\PY@it=\textit\def\PY@tc##1{\textcolor[rgb]{0.25,0.50,0.50}{##1}}}
\expandafter\def\csname PY@tok@bp\endcsname{\def\PY@tc##1{\textcolor[rgb]{0.00,0.50,0.00}{##1}}}
\expandafter\def\csname PY@tok@kc\endcsname{\let\PY@bf=\textbf\def\PY@tc##1{\textcolor[rgb]{0.00,0.50,0.00}{##1}}}
\expandafter\def\csname PY@tok@sd\endcsname{\let\PY@it=\textit\def\PY@tc##1{\textcolor[rgb]{0.73,0.13,0.13}{##1}}}
\expandafter\def\csname PY@tok@gp\endcsname{\let\PY@bf=\textbf\def\PY@tc##1{\textcolor[rgb]{0.00,0.00,0.50}{##1}}}
\expandafter\def\csname PY@tok@gs\endcsname{\let\PY@bf=\textbf}
\expandafter\def\csname PY@tok@gu\endcsname{\let\PY@bf=\textbf\def\PY@tc##1{\textcolor[rgb]{0.50,0.00,0.50}{##1}}}
\expandafter\def\csname PY@tok@kp\endcsname{\def\PY@tc##1{\textcolor[rgb]{0.00,0.50,0.00}{##1}}}
\expandafter\def\csname PY@tok@dl\endcsname{\def\PY@tc##1{\textcolor[rgb]{0.73,0.13,0.13}{##1}}}
\expandafter\def\csname PY@tok@k\endcsname{\let\PY@bf=\textbf\def\PY@tc##1{\textcolor[rgb]{0.00,0.50,0.00}{##1}}}
\expandafter\def\csname PY@tok@vg\endcsname{\def\PY@tc##1{\textcolor[rgb]{0.10,0.09,0.49}{##1}}}
\expandafter\def\csname PY@tok@ss\endcsname{\def\PY@tc##1{\textcolor[rgb]{0.10,0.09,0.49}{##1}}}
\expandafter\def\csname PY@tok@m\endcsname{\def\PY@tc##1{\textcolor[rgb]{0.40,0.40,0.40}{##1}}}
\expandafter\def\csname PY@tok@na\endcsname{\def\PY@tc##1{\textcolor[rgb]{0.49,0.56,0.16}{##1}}}
\expandafter\def\csname PY@tok@kn\endcsname{\let\PY@bf=\textbf\def\PY@tc##1{\textcolor[rgb]{0.00,0.50,0.00}{##1}}}
\expandafter\def\csname PY@tok@c\endcsname{\let\PY@it=\textit\def\PY@tc##1{\textcolor[rgb]{0.25,0.50,0.50}{##1}}}
\expandafter\def\csname PY@tok@se\endcsname{\let\PY@bf=\textbf\def\PY@tc##1{\textcolor[rgb]{0.73,0.40,0.13}{##1}}}
\expandafter\def\csname PY@tok@ni\endcsname{\let\PY@bf=\textbf\def\PY@tc##1{\textcolor[rgb]{0.60,0.60,0.60}{##1}}}
\expandafter\def\csname PY@tok@si\endcsname{\let\PY@bf=\textbf\def\PY@tc##1{\textcolor[rgb]{0.73,0.40,0.53}{##1}}}
\expandafter\def\csname PY@tok@o\endcsname{\def\PY@tc##1{\textcolor[rgb]{0.40,0.40,0.40}{##1}}}
\expandafter\def\csname PY@tok@nv\endcsname{\def\PY@tc##1{\textcolor[rgb]{0.10,0.09,0.49}{##1}}}
\expandafter\def\csname PY@tok@cp\endcsname{\def\PY@tc##1{\textcolor[rgb]{0.74,0.48,0.00}{##1}}}
\expandafter\def\csname PY@tok@kr\endcsname{\let\PY@bf=\textbf\def\PY@tc##1{\textcolor[rgb]{0.00,0.50,0.00}{##1}}}
\expandafter\def\csname PY@tok@s\endcsname{\def\PY@tc##1{\textcolor[rgb]{0.73,0.13,0.13}{##1}}}
\expandafter\def\csname PY@tok@kt\endcsname{\def\PY@tc##1{\textcolor[rgb]{0.69,0.00,0.25}{##1}}}
\expandafter\def\csname PY@tok@sx\endcsname{\def\PY@tc##1{\textcolor[rgb]{0.00,0.50,0.00}{##1}}}
\expandafter\def\csname PY@tok@s2\endcsname{\def\PY@tc##1{\textcolor[rgb]{0.73,0.13,0.13}{##1}}}
\expandafter\def\csname PY@tok@gd\endcsname{\def\PY@tc##1{\textcolor[rgb]{0.63,0.00,0.00}{##1}}}

\def\PYZbs{\char`\\}
\def\PYZus{\char`\_}
\def\PYZob{\char`\{}
\def\PYZcb{\char`\}}
\def\PYZca{\char`\^}
\def\PYZam{\char`\&}
\def\PYZlt{\char`\<}
\def\PYZgt{\char`\>}
\def\PYZsh{\char`\#}
\def\PYZpc{\char`\%}
\def\PYZdl{\char`\$}
\def\PYZhy{\char`\-}
\def\PYZsq{\char`\'}
\def\PYZdq{\char`\"}
\def\PYZti{\char`\~}
% for compatibility with earlier versions
\def\PYZat{@}
\def\PYZlb{[}
\def\PYZrb{]}
\makeatother


    % Exact colors from NB
    \definecolor{incolor}{rgb}{0.0, 0.0, 0.5}
    \definecolor{outcolor}{rgb}{0.545, 0.0, 0.0}



    
    % Prevent overflowing lines due to hard-to-break entities
    \sloppy 
    % Setup hyperref package
    \hypersetup{
      breaklinks=true,  % so long urls are correctly broken across lines
      colorlinks=true,
      urlcolor=urlcolor,
      linkcolor=linkcolor,
      citecolor=citecolor,
      }
    % Slightly bigger margins than the latex defaults
    
    \geometry{verbose,tmargin=1in,bmargin=1in,lmargin=1in,rmargin=1in}
    
    

    \begin{document}
    
    
    \maketitle
    
    

    
    \section{Python科学计算
演示程序使用说明}\label{pythonux79d1ux5b66ux8ba1ux7b97-ux6f14ux793aux7a0bux5e8fux4f7fux7528ux8bf4ux660e}

    \begin{Verbatim}[commandchars=\\\{\}]
{\color{incolor}In [{\color{incolor}1}]:} \PY{k+kn}{import} \PY{n+nn}{subprocess}
        \PY{k+kn}{import} \PY{n+nn}{os}
        \PY{k+kn}{from} \PY{n+nn}{os} \PY{k}{import} \PY{n}{path}
        \PY{k+kn}{import} \PY{n+nn}{re}
        \PY{c+c1}{\PYZsh{}from IPython.nbformat import read}
        \PY{k+kn}{from} \PY{n+nn}{nbformat} \PY{k}{import} \PY{n}{read}
\end{Verbatim}


    本文件夹保存本书所有章节对应的IPython
Notebook文件。为了正确显示其中的SVG图表,需要运行下面的程序``Trust''所有的Notebook:

    \begin{Verbatim}[commandchars=\\\{\}]
{\color{incolor}In [{\color{incolor}2}]:} \PY{k}{for} \PY{n}{folder}\PY{p}{,} \PY{n}{subfolders}\PY{p}{,} \PY{n}{filenames} \PY{o+ow}{in} \PY{n}{os}\PY{o}{.}\PY{n}{walk}\PY{p}{(}\PY{l+s+s2}{\PYZdq{}}\PY{l+s+s2}{.}\PY{l+s+s2}{\PYZdq{}}\PY{p}{)}\PY{p}{:}
            \PY{k}{for} \PY{n}{filename} \PY{o+ow}{in} \PY{n}{filenames}\PY{p}{:}
                \PY{n}{fullpath} \PY{o}{=} \PY{n}{path}\PY{o}{.}\PY{n}{join}\PY{p}{(}\PY{n}{folder}\PY{p}{,} \PY{n}{filename}\PY{p}{)}
                \PY{k}{if} \PY{n}{fullpath}\PY{o}{.}\PY{n}{lower}\PY{p}{(}\PY{p}{)}\PY{o}{.}\PY{n}{endswith}\PY{p}{(}\PY{l+s+s2}{\PYZdq{}}\PY{l+s+s2}{.ipynb}\PY{l+s+s2}{\PYZdq{}}\PY{p}{)}\PY{p}{:}
                    \PY{c+c1}{\PYZsh{}subprocess.call([\PYZdq{}jupyter\PYZdq{}, \PYZdq{}trust\PYZdq{}, fullpath, \PYZdq{}\PYZhy{}\PYZhy{}profile=scipybook2\PYZdq{}])}
                    \PY{n}{subprocess}\PY{o}{.}\PY{n}{call}\PY{p}{(}\PY{p}{[}\PY{l+s+s2}{\PYZdq{}}\PY{l+s+s2}{jupyter}\PY{l+s+s2}{\PYZdq{}}\PY{p}{,} \PY{l+s+s2}{\PYZdq{}}\PY{l+s+s2}{trust}\PY{l+s+s2}{\PYZdq{}}\PY{p}{,} \PY{n}{fullpath}\PY{p}{]}\PY{p}{)}
\end{Verbatim}


    \begin{itemize}
\item
  通过\url{examples.ipynb}可以运行本书提供的所有实例程序。
\item
  通过\href{../../nbextensions/}{nbextensions}可以开关Notebook的所有Javascript插件。
\item
  本书采用Notebook编写,请打开\href{01-intro/notebook-train.ipynb}{IPython
  Notebook操作练习}学习Notebook的基本操作。
\item
  请打开\href{01-intro/scpy2-magics.ipynb}{本书提供的魔法命令}查看本书新增的所有魔法命令。
\item
  运行下面的程序可以得到所有章节对应的Notebook文件链接:
\end{itemize}

    \begin{Verbatim}[commandchars=\\\{\}]
{\color{incolor}In [{\color{incolor}21}]:} \PY{n}{links} \PY{o}{=} \PY{p}{[}\PY{p}{]}
         \PY{k}{for} \PY{n}{folder}\PY{p}{,} \PY{n}{\PYZus{}}\PY{p}{,} \PY{n}{filenames} \PY{o+ow}{in} \PY{n}{os}\PY{o}{.}\PY{n}{walk}\PY{p}{(}\PY{l+s+s2}{\PYZdq{}}\PY{l+s+s2}{.}\PY{l+s+s2}{\PYZdq{}}\PY{p}{)}\PY{p}{:}
             \PY{k}{for} \PY{n}{filename} \PY{o+ow}{in} \PY{n}{filenames}\PY{p}{:}
                 \PY{c+c1}{\PYZsh{}if re.match(r\PYZdq{}\PYZbs{}w+\PYZhy{}[0\PYZhy{}9a\PYZhy{}zA\PYZhy{}Z]\PYZbs{}d\PYZbs{}d\PYZhy{}.+?\PYZbs{}.ipynb\PYZdl{}\PYZdq{}, filename):}
                 \PY{k}{if} \PY{n}{re}\PY{o}{.}\PY{n}{match}\PY{p}{(}\PY{l+s+sa}{r}\PY{l+s+s2}{\PYZdq{}}\PY{l+s+s2}{\PYZbs{}}\PY{l+s+s2}{w+\PYZhy{}[0\PYZhy{}9]}\PY{l+s+s2}{\PYZbs{}}\PY{l+s+s2}{d}\PY{l+s+s2}{\PYZbs{}}\PY{l+s+s2}{d\PYZhy{}+}\PY{l+s+s2}{\PYZbs{}}\PY{l+s+s2}{w+}\PY{l+s+s2}{\PYZbs{}}\PY{l+s+s2}{.ipynb\PYZdl{}}\PY{l+s+s2}{\PYZdq{}}\PY{p}{,} \PY{n}{filename}\PY{p}{)}\PY{p}{:}
                     \PY{n}{fullpath} \PY{o}{=} \PY{n}{path}\PY{o}{.}\PY{n}{join}\PY{p}{(}\PY{n}{folder}\PY{p}{,} \PY{n}{filename}\PY{p}{)}
                     \PY{n}{book} \PY{o}{=} \PY{n}{read}\PY{p}{(}\PY{n}{fullpath}\PY{p}{,} \PY{l+m+mi}{4}\PY{p}{)}
                     \PY{k}{for} \PY{n}{cell} \PY{o+ow}{in} \PY{n}{book}\PY{o}{.}\PY{n}{cells}\PY{p}{:}
                         \PY{k}{if} \PY{n}{cell}\PY{o}{.}\PY{n}{cell\PYZus{}type} \PY{o}{==} \PY{l+s+s2}{\PYZdq{}}\PY{l+s+s2}{markdown}\PY{l+s+s2}{\PYZdq{}} \PY{o+ow}{and} \PY{n}{cell}\PY{o}{.}\PY{n}{source}\PY{o}{.}\PY{n}{startswith}\PY{p}{(}\PY{l+s+s2}{\PYZdq{}}\PY{l+s+s2}{\PYZsh{}}\PY{l+s+s2}{\PYZdq{}}\PY{p}{)}\PY{p}{:}
                             \PY{n}{title} \PY{o}{=} \PY{n}{cell}\PY{o}{.}\PY{n}{source}\PY{o}{.}\PY{n}{strip}\PY{p}{(}\PY{l+s+s2}{\PYZdq{}}\PY{l+s+s2}{\PYZsh{} }\PY{l+s+s2}{\PYZdq{}}\PY{p}{)}
                             \PY{n}{name} \PY{o}{=} \PY{n}{path}\PY{o}{.}\PY{n}{splitext}\PY{p}{(}\PY{n}{filename}\PY{p}{)}\PY{p}{[}\PY{l+m+mi}{0}\PY{p}{]}
                             \PY{n}{folder} \PY{o}{=} \PY{n}{path}\PY{o}{.}\PY{n}{basename}\PY{p}{(}\PY{n}{folder}\PY{p}{)}
                             \PY{n}{link} \PY{o}{=} \PY{l+s+s2}{\PYZdq{}}\PY{l+s+s2}{[}\PY{l+s+si}{\PYZob{}title\PYZcb{}}\PY{l+s+s2}{ \PYZhy{} }\PY{l+s+si}{\PYZob{}name\PYZcb{}}\PY{l+s+s2}{](}\PY{l+s+si}{\PYZob{}folder\PYZcb{}}\PY{l+s+s2}{/}\PY{l+s+si}{\PYZob{}name\PYZcb{}}\PY{l+s+s2}{.ipynb)}\PY{l+s+s2}{\PYZdq{}}\PY{o}{.}\PY{n}{format}\PY{p}{(}
                                 \PY{n}{title}\PY{o}{=}\PY{n}{title}\PY{p}{,} \PY{n}{name}\PY{o}{=}\PY{n}{name}\PY{p}{,} \PY{n}{folder}\PY{o}{=}\PY{n}{folder}\PY{p}{)}
                             \PY{n}{links}\PY{o}{.}\PY{n}{append}\PY{p}{(}\PY{n}{link}\PY{p}{)}
                             \PY{k}{break}
         
         \PY{k+kn}{from} \PY{n+nn}{IPython}\PY{n+nn}{.}\PY{n+nn}{display} \PY{k}{import} \PY{n}{display\PYZus{}markdown}\PY{p}{,} \PY{n}{Markdown}
         \PY{n}{display\PYZus{}markdown}\PY{p}{(}\PY{n}{Markdown}\PY{p}{(}\PY{l+s+s2}{\PYZdq{}}\PY{l+s+se}{\PYZbs{}n}\PY{l+s+se}{\PYZbs{}n}\PY{l+s+s2}{\PYZdq{}}\PY{o}{.}\PY{n}{join}\PY{p}{(}\PY{n}{links}\PY{p}{)}\PY{p}{)}\PY{p}{)}
\end{Verbatim}


    \href{01-intro/intro-100-whypython.ipynb}{Python科学计算环境的安装与简介
- intro-100-whypython}

\href{01-intro/intro-200-ipython.ipynb}{IPython Notebook入门 -
intro-200-ipython}

\href{01-intro/intro-300-library.ipynb}{扩展库介绍 - intro-300-library}

\href{02-numpy/numpy-100-ndarray.ipynb}{NumPy-快速处理数据 -
numpy-100-ndarray}

\href{02-numpy/numpy-200-ufunc.ipynb}{ufunc函数 - numpy-200-ufunc}

\href{02-numpy/numpy-300-mulitindex.ipynb}{多维数组的下标存取 -
numpy-300-mulitindex}

\href{02-numpy/numpy-400-functions.ipynb}{庞大的函数库 -
numpy-400-functions}

\href{02-numpy/numpy-470-gufuncs.ipynb}{广义ufunc函数 -
numpy-470-gufuncs}

\href{02-numpy/numpy-900-tips.ipynb}{实用技巧 - numpy-900-tips}

\href{03-scipy/scipy-100-intro.ipynb}{SciPy-数值计算库 -
scipy-100-intro}

\href{03-scipy/scipy-210-optimize.ipynb}{拟合与优化-optimize -
scipy-210-optimize}

\href{03-scipy/scipy-310-linalg.ipynb}{线性代数-linalg -
scipy-310-linalg}

\href{03-scipy/scipy-400-stats.ipynb}{统计-stats - scipy-400-stats}

\href{03-scipy/scipy-500-integrate.ipynb}{数值积分-integrate -
scipy-500-integrate}

\href{03-scipy/scipy-600-signal.ipynb}{信号处理-signal -
scipy-600-signal}

\href{03-scipy/scipy-700-interpolate.ipynb}{插值-interpolate -
scipy-700-interpolate}

\href{03-scipy/scipy-810-sparse.ipynb}{稀疏矩阵-sparse -
scipy-810-sparse}

\href{03-scipy/scipy-900-ndimage.ipynb}{图像处理-ndimage -
scipy-900-ndimage}

\href{04-matplotlib/matplotlib-100-fastdraw.ipynb}{matplotlib-绘制精美的图表
- matplotlib-100-fastdraw}

\href{04-matplotlib/matplotlib-200-artists.ipynb}{Artist对象 -
matplotlib-200-artists}

\href{04-matplotlib/matplotlib-300-transform.ipynb}{坐标变换和注释 -
matplotlib-300-transform}

\href{04-matplotlib/matplotlib-600-tips.ipynb}{matplotlib技巧集 -
matplotlib-600-tips}

\href{05-pandas/pandas-100-dataobjects.ipynb}{Pandas-方便的数据分析库 -
pandas-100-dataobjects}

\href{05-pandas/pandas-200-getset.ipynb}{下标存取 - pandas-200-getset}

\href{05-pandas/pandas-300-io.ipynb}{文件的输入输出 - pandas-300-io}

\href{05-pandas/pandas-400-calculation.ipynb}{数值运算函数 -
pandas-400-calculation}

\href{05-pandas/pandas-500-string.ipynb}{字符串处理 - pandas-500-string}

\href{05-pandas/pandas-600-datetime.ipynb}{时间序列 -
pandas-600-datetime}

\href{05-pandas/pandas-700-nan.ipynb}{与\texttt{NaN}相关的函数 -
pandas-700-nan}

\href{05-pandas/pandas-800-changeshape.ipynb}{改变DataFrame的形状 -
pandas-800-changeshape}

\href{05-pandas/pandas-900-groupby.ipynb}{分组运算 - pandas-900-groupby}

\href{06-sympy/sympy-100-intro.ipynb}{SymPy-符号运算好帮手 -
sympy-100-intro}

\href{06-sympy/sympy-200-expression.ipynb}{数学表达式 -
sympy-200-expression}

\href{06-sympy/sympy-300-calculations.ipynb}{符号运算 -
sympy-300-calculations}

\href{06-sympy/sympy-400-output.ipynb}{输出符号表达式 -
sympy-400-output}

\href{06-sympy/sympy-500-mechanics.ipynb}{机械运动模拟 -
sympy-500-mechanics}

\href{07-traits/traits-100-intro.ipynb}{Traits \&
TraitsUI-轻松制作图形界面 - traits-100-intro}

\href{07-traits/traits-200-types.ipynb}{Trait类型 - traits-200-types}

\href{07-traits/traits-300-uiintro.ipynb}{TraitsUI入门 -
traits-300-uiintro}

\href{07-traits/traits-400-handler.ipynb}{用Handler控制界面和模型 -
traits-400-handler}

\href{07-traits/traits-500-editors.ipynb}{属性编辑器 -
traits-500-editors}

\href{07-traits/traits-600-example.ipynb}{函数曲线绘制工具 -
traits-600-example}

\href{08-tvtk_mayavi/tvtk_mayavi-100-intro.ipynb}{TVTK与Mayavi-数据的三维可视化
- tvtk\_mayavi-100-intro}

\href{08-tvtk_mayavi/tvtk_mayavi-200-pipeline.ipynb}{VTK的流水线(Pipeline)
- tvtk\_mayavi-200-pipeline}

\href{08-tvtk_mayavi/tvtk_mayavi-300-dataset.ipynb}{数据集 -
tvtk\_mayavi-300-dataset}

\href{08-tvtk_mayavi/tvtk_mayavi-400-tvtk_and_vtk.ipynb}{TVTK的改进 -
tvtk\_mayavi-400-tvtk\_and\_vtk}

\href{08-tvtk_mayavi/tvtk_mayavi-600-mlab.ipynb}{用mlab快速绘图 -
tvtk\_mayavi-600-mlab}

\href{09-opencv/opencv-200-imgprocess.ipynb}{图像处理 -
opencv-200-imgprocess}

\href{09-opencv/opencv-300-transforms.ipynb}{图像变换 -
opencv-300-transforms}

\href{09-opencv/opencv-400-identify.ipynb}{图像识别 -
opencv-400-identify}

\href{09-opencv/opencv-500-shapes.ipynb}{形状与结构分析 -
opencv-500-shapes}

\href{10-cython/cython-100-compiler.ipynb}{Cython-编译Python程序 -
cython-100-compiler}

\href{10-cython/cython-200-intro.ipynb}{Cython入门 - cython-200-intro}

\href{10-cython/cython-300-memoryview.ipynb}{高效处理数组 -
cython-300-memoryview}

\href{10-cython/cython-600-tips.ipynb}{Cython技巧集 - cython-600-tips}

\href{11-examples/examples-000-intro.ipynb}{实例 - examples-000-intro}

\href{11-examples/examples-100-possion.ipynb}{使用泊松混合合成图像 -
examples-100-possion}

\href{11-examples/examples-300-movielens.ipynb}{推荐算法 -
examples-300-movielens}

\href{11-examples/examples-400-fft.ipynb}{频域信号处理 -
examples-400-fft}

\href{11-examples/examples-500-picosat.ipynb}{布尔可满足性问题求解器 -
examples-500-picosat}

\href{11-examples/examples-600-fractal.ipynb}{分形 -
examples-600-fractal}

    
    \begin{Verbatim}[commandchars=\\\{\}]
{\color{incolor}In [{\color{incolor} }]:} 
\end{Verbatim}



    % Add a bibliography block to the postdoc
    
    
    
    \end{document}
